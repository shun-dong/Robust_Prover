A \emph{floor function} $\lfloor \cdot \rfloor : R \to \Z$ on a totally ordered ring $R$ is a function such that, for any $r \in R$ and $n \in \Z$, $n \leq \lfloor r \rfloor$ if and only if $n \leq r$ in $R$.

Let $R$ and $S$ be totally ordered rings with floor.
Find all functions $f : R \to S$ such that, for any $x, y \in R$,
\[ f(\lfloor x \rfloor y) = f(x) \lfloor f(y) \rfloor. \tag{*}\label{2010a1-eq0} \]



\subsection*{Answer}

One can show that an ordered ring $R$ with floor is either dense or isomorphic to $\Z$.
By dense, we mean that for any $x < y$ in $R$, there exists $z \in R$ such that $x < z < y$.

If $R$ is dense, then the only functions satisfying~\eqref{2010a1-eq0} are $f \equiv 0$ or $f \equiv C$ for some $C \in S$ with $\lfloor C \rfloor = 1$.

If $R = \Z$, then one of the following holds:
\begin{itemize}

    \item
    $f$ is induced from a monoid homomorphism $g : \Z \to \Z$.
    
    \item
    There exists a monoid homomorphism $\phi : \Z \to \N$ and an positive infinitesimal element $\epsilon \in R$ such that $f(n) = (1 + \epsilon) \phi(n)$ for all $n \in \Z$.
    Here, infinitesimal means that $k |\epsilon| < 1$ for all $k \in \N$.
    Since $\epsilon$ is positive, this is just saying that $k \epsilon < 1$ for all $k \in \N$.

    \item
    There exists a set $A \subseteq \Z$ with the property:
    (1). for any $m, n \in \Z$, we have $mn \in A \iff m \in A \wedge n \in A$;
    (2). for any $n \in \Z$, we have $f(n) = C$ if $n \in A$ and $f(n) = 0$ otherwise.

\end{itemize}





\subsection*{Solution ($R$ is densely ordered)}

Plugging $(x, y) = (0, 0)$ into~\eqref{2010a1-eq0} yields $f(0) = f(0) \lfloor f(0) \rfloor$.
Thus either $\lfloor f(0) \rfloor = 1$ or $f(0) = 0$.
In the former case, plugging $y = 0$ into~\eqref{2010a1-eq0} yields that $f$ is constant.
Clearly the underlying constant $C$ is equal to $f(0)$ and we have obtained $\lfloor C \rfloor = 1$.
We now assume that $f(0) = 0$ and show that $f \equiv 0$.

Since $R$ is densely ordered, there exists $c \in R$ such that $0 < c < 1$.
Plugging $x = y = c$ into~\eqref{2010a1-eq0} yields $f(c) \lfloor f(c) \rfloor = f(0) = 0$, and thus $\lfloor f(c) \rfloor = 0$.
Plugging $x = -1$ and $y = c$ then yields $f(-c) = 0$.
Finally, since $\lfloor -c \rfloor = -1$, plugging $x = -c$ and replacing $y$ with $-y$ yields $f \equiv 0$.





\subsection*{Solution ($R = \Z$)}

Here,~\eqref{2010a1-eq0} simplifies as $f(xy) = f(x) \lfloor f(y) \rfloor$.
Plugging $y = 1$ yields either $f \equiv 0$ or $\lfloor f(1) \rfloor = 1$.
From now on, we assume that $\lfloor f(1) \rfloor = 1$.

Let $\varepsilon = f(1) - 1$ and $g(n) = \lfloor f(n) \rfloor$ for any $n \in \Z$.
Plugging $x = 1$ yields
\[ f(y) = f(1) g(y) \quad \forall y \in \Z. \tag{1}\label{2010a1-eq1} \]
In particular, we get
\[ 0 \leq \varepsilon g(y) < 1 \quad \forall y \in \Z. \tag{2}\label{2010a1-eq2} \]
Next,~\eqref{2010a1-eq0} yields
\[ f(1) g(xy) = f(1) g(x) g(y) \implies g(xy) = g(x) g(y) \quad \forall x, y \in \Z, \]
    since $g(1) = \lfloor f(1) \rfloor = 1$ implies $f(1) \neq 0$.
Thus, $g$ is a multiplicative function.

If $g(0) \neq 0$, it is easy to see that $g \equiv 1$, so $f$ is constant.
So now, we assume that $g(0) = 0$.
Since $g(1) = 1$,~\eqref{2010a1-eq2} yields $0 \leq \varepsilon < 1$.
If $\varepsilon = 0$, then $f$ maps integers to integers and is a multiplicative function.
From now on, we assume that $0 < \varepsilon < 1$.

If $\varepsilon > 0$, then~\eqref{2010a1-eq2} means that $g$ only attains non-negative values.
If $g$ attains a value greater than $1$, then $g$ is unbounded from above.
Thus~\eqref{2010a1-eq2} yields $0 < k \varepsilon < 1$ for any positive integer $k$.
This implies that $\varepsilon$ is infinitesimal.

Finally, suppose that $g(0) = 0$, $g(1) = 1$, and $g(n) \in \{0, 1\}$ for any $n \in \Z$.
Then we take $A = g^{-1}(1)$; it remains to show that $mn \in A \iff m \in A \wedge n \in A$.
This is easy since $g(mn) = g(m) g(n)$ for all $m, n \in \Z$.





\subsection*{Extra notes}

The result for the case $R = \Z$ is indeed the best we can do.
Each of the three non-trivial cases has at least one non-trivial example.
In the first case, for example, we can take $g(n) = c^{\nu_2(n)}$ for some integer $c$ for $n \neq 0$ (and $g(0) = 0$).
In the second case, the set $M$ that we can choose are precisely $\Z$ and those defined as the set of integers whose prime factors belong to a fixed subset of the primes.
In the third case, there exists a floor ring $S$ with a non-zero infinitesimal element $\epsilon$.
Indeed, take $S = \Z[X]$ with the ordering defined by $f(X) = a_0 + a_1 X + \ldots + a_n X^n > 0$ if and only if $a_i > 0$, where $i$ is the smallest index such that $a_i \neq 0$.
One can check that $S$ is a totally ordered ring and for any $f \in \Z[X]$, we can choose a unique $n \in \Z$ such that $n \leq f < n + 1$.
Namely, $n = f(0)$ if $f(X) - f(0) \geq 0$, and $n = f(0) - 1$ if $f(X) - f(0) < 0$.
