Let $f : \N^+ \to \N^+$ be a function such that $f(m + n) \mid f(m) + f(n)$ for all $m, n \in \N^+$.
Prove that there exists $n \in \N^+$ such that $f(n) \mid f(n)$ for all $n \in \N^+$.



\subsection*{Solution}

We only use the first lemma and a bit of similar idea for the unbounded case from the official solution.
Afterwards, we proceed on our own, and we prove more properties.
Specifically, we show that one of the following holds:
\begin{itemize}
    \item   $f(n) = f(1) n$ for all $n \in \N^+$;
    \item   $f$ is eventually constant;
    \item   there exists $d > 1$ and $N, c \in \N^+$ such that for all $n \geq N$,
            \[ f(n) = \begin{cases} 2c, & d \mid n, \\ c, & d \nmid n. \end{cases} \]
\end{itemize}
In the second case, if $f(n) = c$ for all $n$ big enough, then we claim that $c \mid f(n)$ for all $n$.
In the third case, we also claim that $c \mid f(n)$ for all $n$.
Both are easy to prove via downwards induction or via the following claim.

\begin{claim}
For any $c \in \N^+$, if the set $S = \{n \in \N^+ : c \mid f(n)\}$ is infinite, then it is of form $d \N^+ = \{nd : n \in \N^+\}$ for some $d \in \N^+$. 
\end{claim}
\begin{proof}
Notice that for any $m, n \in \N^+$, if $m, m + n \in S$, then $n \in S$, which follows from the given condition on $f$.
By downwards induction, for any $m, n \in S$ and $k \leq n$ with $k \equiv n \pmod{m}$, we also have $k \in S$.

Now take $d = \min(S)$ and claim that $S \subseteq d \N^+$.
Otherwise, for some $n \in S$, we get $n \% d \in S$ with $0 < n \% d < d$, where $n \% d$ is the remainder of $n$ divided by $d$.
Conversely, for any $n$, since $S$ is infinite, there exists $N \geq n$ such that $Nd \in S$.
Then the first paragraph yields $nd \in S$, which implies $d \N^+ \subseteq S$ and thus $S = d \N^+$.
\end{proof}

First, consider the case where $f$ is unbounded, and let $a = f(1)$.
By induction, it is easy to see that $f(n) \leq an$ for all $n \in \N^+$.
We now claim that $f(n) = an$ for all $n \in \N^+$.
By using induction, it suffices to show that $f(n + 1) = f(n) + f(1)$ for all $n \in \N^+$.

Let $N$ be the minimal positive integer satisfying $f(N) > a(2n + 1)$.
Since $f(N) \leq aN$, we have $N > 2n + 1$.
By minimality, we have $f(m) \leq a(2n + 1) < f(N)$ for all $m < N$.
In particular, we get $f(N) \mid f(m) + f(N - m) < 2 f(N)$, so $f(m) + f(N - m) = f(N)$ for each $m < N$.
From this, we get
\[ an = f(n) + f(N - n) = f(n + 1) + f(N - n - 1). \]
It remains to show that $f(N - n) = f(N - n - 1) + a$.
Indeed, we know that $f(N - n) \mid f(N - n - 1) + a$.
On the other hand, $f(N - n) + f(n) \geq f(N) > a(2n + 1)$, so $f(N - n) > a(n + 1)$.
Also, minimality yields $f(N - n - 1) + a \leq a(2n + 1) + a = 2a(n + 1)$.
Thus $2 f(N - n) > f(N - n - 1) + a$, which yields $f(N - n) = f(N - n - 1) + a$, as desired.

Now we consider the case where $f$ is bounded.
Then there exists at least one value of $f$ that is attained infinitely many times.
Let $M$ be the maximal such value, and pick some $N$ such that $f(n) \leq M$ for all $n \geq N$.
By our choice of $M$, equality happens infinitely many times.
Using the first claim on $\{n \in \N^+ : M \mid f(n)\}$, there exists $d \in \N^+$ such that $M \mid f(n) \iff d \mid n$.
For $n \geq N$, this yields $f(n) = M \iff d \mid n$.
If $d = 1$, then indeed $f$ is eventually constant, so now assume that $d > 1$.
If $d > 1$, we claim that $M$ is even and $f(n) = M/2$ for all $n \geq N$ such that $d \nmid n$.

Fix some $k, m \geq N$ such that $d \mid k + m$ and $d \nmid k$.
Then $f(k + m) = M > f(k), f(m)$, we get $f(k) + f(m) = M$.
The same holds with $k$ replaced by $k + Nd$, so
\[ f(k) + f(m) = f(k + Nd) + f(m) = M \implies f(k) = f(k + Nd). \]
Since $Nd \geq N$, we have $f(Nd) = M$, so the above yields
\[ f(k) = f(k + Nd) \mid f(k) + M \iff f(k) \mid M = f(k) + f(m) \iff f(k) \mid f(m). \]
By symmetry, $f(m) \mid f(k)$ as well, yielding $f(k) = f(m)$.
Since $2 f(k) = f(k) + f(m) = M$, we get $f(k) = M/2$.
We get $f(n) = M/2$ for all $n \geq N$ with $d \mid n$ by taking $k = n$ and $M = n(d - 1)$.



\subsection*{Extra notes}

On the other hand, it is quite obvious that any one of the three conditions "almost" suffice to imply that $f$ satisfies the given property.
The function $f(n) = an$ and any constant function works.
The periodic function $f$ defined by $f(n) = 2$ if $d \mid n$ and $f(n) = 1$ otherwise also satisfy the given condition.
It is not hard either to modify the first few terms of all these functions to get another function that satisfies the given condition.
